\documentclass[a4paper]{article}

\usepackage[english]{babel}
\usepackage[utf8]{inputenc}
\usepackage{algorithm}
\usepackage{algorithmicx}
\usepackage[noend]{algpseudocode}
\usepackage{mathtools}

\title{NLP - Master Notes}

\algdef{SE}[DOWHILE]{Do}{doWhile}{\algorithmicdo}[1]{\algorithmicwhile\ #1}%

\begin{document}
	\maketitle
	
	\section{Language Models}
	
	\subsection{N-Gram Models}
	
	\begin{itemize}
	
		\item Language (prediction) models which make the \textit{Markov assumption} for an $(n-1)^{th}$ order Markov Model; i.e. that only the previous n-1 words have a probabilistic dependence on the current word.
		
		\item Probability of words 1 to n: $P(w_1^n) = \prod_{k=1}^{n}P(w_k | w_{k-N+1}^{k-1})$
		
		General steps for creating an n-gram model:		
		\begin{enumerate}
			\item choose a vocabulary 
			\item add $<s>$ and $</s$ symbols 
			\item replace unknown words in the training corpus with $<UNK>$ 
			\item calculate probabilities (on an as needs basis?)
			\item calculate most probably words in order (until reaching end of sentence symbol) OR evaluate perplexity of test corpus using above formulas.
			
		\end{enumerate}
		
		\item \textit{example:} 
		
		\begin{center}
			Text: \\
			One cat sat. Three \textbf{cats sat}. Eight \textbf{cats sat}. The \textbf{cats} had nine lives. 	\\
			
			P(sat $|$ cats sat) = $\frac{C(cats,sat)}{C(cat)} = \frac{2}{3}$
			
		\end{center}
		
		\item \textbf{Definition of a language model}: a model which assigns probabilities to sentences, based on the training corpus. The sum of all the probabilities of all possible sentences (of arbitrary length) should equal 1. \\ 
			
			Trivial example:
			\begin{itemize}
				\item Training corpus contains two sentences: \\ 
				1. ``a b", \\ 
				2. ``b a"
				
				\item Append $<sos>$ and $</sos>$ to each: \\
				1. ``s a b /s" 
				\\ 2. ``s b a /s"
				
				\item generate the probabilities of a bigram (N=2) language model: \\
				
				$P(a | s) = \frac{1}{2}$ \\
				$P(b | s) = \frac{1}{2}$ \\
				$P(b | a) = \frac{1}{2}$ \\
				$P(a | b) = \frac{1}{2}$ \\
				$P(/s | a) = \frac{1}{2}$ \\
				$P(/s | b) = \frac{1}{2}$ \\
				
				\item To calculate the probability of ALL possible sentences, we take the prob of all sentences of length 1, all sentences of length 2, etc. Note when calculating this, the TEST sentences need to include $<s> and </s>$
				
				\item For example: P(a) = P($<s> a </s>$) = $P(a | <s>) * P(</s> | a)$ = 1/4
				
				\item Probability is same for b, so the sum of all probabilities of sentence length 1 = $\frac{1}{2}$
				
				The sum of all sentences will be the infinite series: \\
				
				$$P(all) = \frac{1}{2} + \sum_{i=2}^{\infty} \frac{i!}{(i-2)!} \frac{1}{2} ^ {i + 1}$$ \\
				
				Does this sum to 1? A proof would be cool.
				
			\end{itemize}
		
		\item Sentence Generation: until you produce a $</s>$ symbol, continually generate words using: $argmax_(w_k) \frac{C(w_{k-n+1},...,w_k)}{C(w_{k-n+1},...,w_{k-1})}$ 
		
		\item Perplexity: how well a model fits the data $PP(W) = P(w1,...w_N)^{-\frac{1}{N}}$, for N words in the test corpus. A perplexity of 1 would be the lowest possible.
		
		\item Smoothing: 
		
			\begin{itemize}
				\item Laplace (add-one): $P(W_n | w_{n - N + 1}^{n-1}) = \frac{C( w_{n - N + 1}^{n-1}w) + 1}{c( w_{n - N + 1}^{n-1}) + V}$, where V is the vocabulary size
				
				\item add $\delta$, normalise by $\delta V$
			
			\end{itemize}
		
	\item Interpolation: creating a linear combination of n-gram models of varying n: $\hat{P}(w_i | w_{i-1},...w_{i-n}) = \sum_{j=2}^{n} \lambda_j * P(w_i | w_{i-1},...,w_{i-1-j})$ , where $\sum_{j}\lambda_j = 1$ 
	
	\item Back-off: back-off to lower n models until data is available (does this mean you can have arbitrary n?)
	
	
		
	\end{itemize}
	
	\textbf{}
	
	
	
\end{document}